\newbox\charbox % Used to meaśure character width
\newdimen\charwidth
\newdimen\charheight

\def\dofont#1, #2!{"#1" at #2\relax}
\newdimen\serifsize
\newdimen\fontsize
\newdimen\scriptsize
\newdimen\scriptscriptsize
% \serifsize is to be set in config, other three are temporary

\newfam\doublefam
\def\mathbb{\fam=\doublefam}

% Switches to monospace font with given size
% Before calling, \monofont must be defined
\def\mkmonofam#1{%
	\fontsize=#1
	% multiplied to visually match serif
	\multiply\fontsize by6
	\divide\fontsize by7
	\font\tenrm=\monofont, \fontsize!%
	\font\tenbf=\monofont/B, \fontsize!%
	\font\tenit=\monofont/I, \fontsize!%
}

% Switches to serif font family with given size.
% Automatically determines \baselineskip, script and superscript sizes.
\def\mkseriffam#1{%
	\fontsize=#1\relax\scriptsize=#1\relax\scriptscriptsize=#1\relax
	\multiply\scriptsize by 7\divide\scriptsize by 10
	\divide\scriptscriptsize by 2
	\font\tenrm=cmr10 at \fontsize\font\sevenrm=cmr7 at \scriptsize\font\fiverm=cmr5 at\scriptscriptsize
	\font\tenbf=cmbx10 at \fontsize
	\font\tenit=cmti10 at \fontsize
	\font\teni=cmmi10 at \fontsize\font\seveni=cmmi7 at \scriptsize\font\fivei=cmmi5 at\scriptscriptsize
	\font\tensy=cmsy10 at \fontsize\font\sevensy=cmsy7 at \scriptsize\font\fivesy=cmsy5 at\scriptscriptsize
	\font\tenex=cmex10 at \fontsize\font\sevenex=cmex7 at \scriptsize\font\fiveex=cmex5 at\scriptscriptsize
	\baselineskip=\fontsize
	\multiply\baselineskip by 12\relax%
	\divide\baselineskip by 10\relax%
	\multiply\fontsize by 6\divide\fontsize by 7
	\font\tentt=\monofont, \fontsize!
	\textfont0=\tenrm \scriptfont0=\sevenrm
	\textfont1=\teni \scriptfont1=\seveni
	\textfont2=\tensy \scriptfont2=\sevensy
	\textfont3=\tenex \scriptfont3=\sevenex
	\font\doubletext=msbm10 at \fontsize
	\textfont\doublefam=\doubletext
}

\def\tt{\tentt}

\input config

\seriffam
\rm

% Scale some of the maths fonts down (use 7pt as a base size, not 10). Not all
% fonts are redefined, but it should be enough for normal usage. Add new
% definitions (\subscriptfont, \*font4 -- \*font15) when needed.
\textfont0=\tenrm \scriptfont0=\sevenrm
\textfont1=\teni \scriptfont1=\seveni
\textfont2=\tensy \scriptfont2=\sevensy
\textfont3=\tenex \scriptfont3=\sevenex

\advance\hoffset by -45pt
\advance\hsize by 90pt
\advance\voffset by -50pt
\advance\vsize by 150pt
\parindent=0pt
\input eplain

% Calculates the character width of the actual font into \spacesize
\def\calcchardimens{%
	\setbox\charbox=\hbox{(}%
	\charwidth=\wd\charbox
	\charheight=\ht\charbox
}

% Defines special escape sequences that print {, }, \ when used in monospace
\def\makeprintable{%
	\def\do##1{\catcode`##1=12}%
	\dospecials
	\catcode`\\=0%
	\catcode`\{=1%
	\catcode`\}=2%
	\gdef\{{\char123\relax}%
	\gdef\}{\char125\relax}%
	\gdef\\##1{\char92\relax##1}% Backspace
}

% Internal macro which sets up the environment for code listing mode
\def\setverb{%
	% Allow blank lines (otherwise multiple consecutive \par's would be treated as one)
	\def\par{\leavevmode\endgraf}%
	%
	% Allow long lines to break, indent line continuations
	\parskip=1.7pt plus .1pt minus .1pt%
	\rightskip=0pt plus 1fill%
	\leftskip=100pt%
	\parindent=-\leftskip%
	\setsimpleverb
}

% Simplified version used for short code snippets
\def\setsimpleverb{%
	\monofam\rm
	\calcchardimens
	\baselineskip=\charheight\relax%
	\makeprintable
	\setspace\obeyspaces\obeylines%
}

% Set up fixed spaces, define tab as four spaces
{%
 \catcode`\^^I=13\catcode`\ =13%
 \gdef\setspace{\catcode`\^^I=13\catcode`\ =13%
 \def {\calcchardimens\hskip\charwidth\relax}\def^^I{    }%
}%
}

% TODO: think of some better delimiter than |, try to make { work (\verb|hello}
% is way less cool than \verb{hello})

% Sets up the verbatim environment, end with \egroup
\def\verb|{%
	\bgroup
	\setverb%
}

% Sets up a simplfied verbatim environment, end with \egroup
\def\simpleverb|{%
	\bgroup
	\setsimpleverb%
}

% \col{rgb color}{font}{content}
\def\col#1#2#3{%
	{%
	\special{color push rgb #1}%
	#2%
	#3%
	\special{color pop}%
	}%
}

% Blank lines: we want to encourage page breaks here
% These macros are inserted automatically by the source code preprocessor
\def\sk{\penalty -500\vskip -.5\charheight plus .5\charheight} % regular
\def\greatsk{\penalty -1000\vskip 0pt plus .5\charheight} % after a function definition (after non-indented "}")

\footline={}
\headline={%
	\vbox{%
		\vskip 5pt

		\headlinefont
		\special{color push rgb 0 0 0}%
		\docname
		\hskip 0pt plus 1filll
		\the\pageno
		\special{color pop}%
		%
		\vskip 3pt
		\hrule height .5pt
	}%
}

% Before the listing of one file
\def\header#1{%
	\writetocentry{notebook}{kapitola|#1}
	{%
	\headerfont
	\vskip 15pt plus 3pt minus 8pt
	\penalty -1000
	#1
	\endgraf
	\penalty 5000
	}%
}

\def\ttheader#1{%
	\header{\ttheaderfont#1}%
}

% Used to print a hash at the beggining of a line
\def\h#1{%
	\calcchardimens\hskip-4\charwidth
	\hcol{#1}%
}

\def\O{{\cal O}}

% TODO: quite a hack
\newbox\usagebox
\newdimen\usagedimen
{
\seriffam\bf
\setbox\usagebox=\hbox{Usage: }
\global\usagedimen=\wd\usagebox
}

% Hotfixes for latex expressions in kactl
\def\text{\hbox}
\def\textrm#1{\hbox{\rm #1}}
\def\texttt#1{\hbox{\tt #1}}
\def\mathtt{}
\def\mod{\bmod}
\def\emph#1{{\it #1\/}}
\def\textbf#1{{\bf #1}}
\def\binom#1#2{{#1\choose #2}}
\def\frac#1#2{{#1\over #2}}

\input headers

\def\img#1#2{%
	\XeTeXpdffile ../src/#2 width #1\relax
}

\def\centerpic#1#2{\par\line{\hss\img{#1}{#2}\hss}\par}
% TODO: Implement text wrapping in \rightpic
%\def\rightpic#1#2{\par\line{\hss\img{#1}{#2}}\par}
\let\rightpic\centerpic
\def\figure#1{\rightpic{.2\hsize}{#1}}


\def\tocnotebookentry#1#2{%
	\tocwork#1|#2|\par
}
\def\tocwork#1|#2|#3|{{%
	\def\tmp{#2}
	\def\tmpB{}
	\ifx\tmp\tmpB
		\def\content{\medskip{\bf #1\hfill #3}}
	\else
		\def\content{{#2}\hfill#3}
	\fi
	\content\par
}}

\def\dotoc{%
	\triplecolumns
	{\let\ttheaderfont\tt\readtocfile}
	\doublecolumns
}
